\documentclass[12pt]{report}


% Title Page
\title{Multi-flow device file}
\author{Di Battista Mattia}
\date{23 03 2022}

\begin{document}
\maketitle

\tableofcontents

\chapter*{Specification}

This specification is related to a Linux device driver implementing low and high priority flows of data.
Through an open session to the device file a thread can read/write data segments. The data delivery
follows a First-in-First-out policy along each of the two different data flows (low and high priority). After
read operations, the read data disappear from the flow. Also, the high priority data flow must offer
synchronous write operations while the low priority data flow must offer an asynchronous execution
(based on delayed work) of write operations, while still keeping the interface able to synchronously notify
the outcome. Read operations are all executed synchronously. The device driver should support 128
devices corresponding to the same amount of minor numbers.

The device driver should implement the support for the \emph{ioctl(..)} service in order to manage the I/O session
as follows:
\begin{itemize}
\item setup of the priority level (high or low) for the operations
\item blocking vs non-blocking read and write operations
\item setup of a timeout regulating the awake of blocking operations
\end{itemize}

A few Linux module parameters and functions should be implemented in order to enable or disable the
device file, in terms of a specific minor number. If it is disabled, any attempt to open a session should fail
(but already open sessions will be still managed). Further additional parameters exposed via VFS should
provide a picture of the current state of the device according to the following information:
\begin{itemize}
\item enabled or disabled
\item number of bytes currently present in the two flows (high vs low priority)
\item number of threads currently waiting for data along the two flows (high vs low priority)
\end{itemize}

\chapter*{title}
\end{document}          
